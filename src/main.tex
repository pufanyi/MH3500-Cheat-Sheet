\documentclass[9pt,landscape]{article}

\usepackage{multicol}

\usepackage{amssymb}
\usepackage{amsthm}
\usepackage{amsmath}
\usepackage{fontspec,xunicode,xltxtra}
\usepackage{titlesec}
\usepackage{indentfirst}
\usepackage{xeCJK}
\usepackage{fancyhdr}
\usepackage{graphicx}
\usepackage{listings}
\usepackage{printlen}
\usepackage{ifthen}
\usepackage[savepos]{zref}
\usepackage{multicol}
\usepackage{sectsty}
\usepackage{xcolor}
\usepackage[framemethod=tikz]{mdframed}
\usepackage{hyperref}

\usepackage[paper=a4paper]{geometry}
\geometry{headheight=2.6cm,headsep=3mm,footskip=13mm}
\geometry{top=2cm,bottom=2cm,left=2cm,right=2cm}


\setCJKmainfont[BoldFont={SimHei}]{SimSun}
\newfontfamily{\monotype}{Consolas}
%\newcommand{\monotype}{\tt}

\pagestyle{fancy}

\fancyhead[L]{MH3500 FINAL CHEAT SHEET}
\fancyhead[R]{}

\setlength{\parindent}{0em}

% settings for listings
\lstset {
  basicstyle = \small\monotype,
  language = C++,
  tabsize = 2,
  breaklines = true,
  breakindent = 1.1em,
  numbers=right,
  stringstyle=\monotype,
  numberstyle=\footnotesize\ttfamily,
  firstnumber=last,
  basewidth={0.5em, 0.4em},
  frame=single
}

% an amazing script
% converts an line-number to arbitrary string
\let\othelstnumber=\thelstnumber
\def\createlinenumber#1#2{
    \edef\thelstnumber{%
        \unexpanded{%
            \ifnum#1=\value{lstnumber}\relax
             \tt #2%
            \else}%
        \expandafter\unexpanded\expandafter{\thelstnumber\othelstnumber\fi}%
    }
    \ifx\othelstnumber=\relax\else
      \let\othelstnumber\relax
    \fi
}

\usepackage{enumitem}

\setlist[itemize]{itemsep=0pt} % formatting template

\begin{document}

\begin{multicols}{3}

\columnseprule=0.25pt

\section{公式}

\subsection{求和}

$\sum_{k=1}^{n}k^3=\frac{1}{4}n^2(n+1)^2$

$\sum_{k=0}^{n-1}r^k=\frac{1-r^n}{1-r}, r\neq 1$

$\sum_{k=1}^{n}kr^k=r\frac{1-(n+1)r^n+nr^{n+1}}{(1-r)^2}, r\neq 1$

\subsection{Maclaurin Series}

$\sin x = \sum_{n=0}^{\infty} \frac{(-1)^n}{(2n+1)!}x^{2n+1} = x - \frac{x^3}{3!} + \frac{x^5}{5!} - \cdots$

$\cos x = \sum_{n=0}^{\infty} \frac{(-1)^n}{(2n)!}x^{2n} = 1 - \frac{x^2}{2!} + \frac{x^4}{4!} - \cdots$

$\ln x = \sum_{n=1}^{\infty} \frac{(-1)^{n-1}}{n}(x-1)^n = (x-1) - \frac{(x-1)^2}{2} - \cdots$

$\frac{1}{1-x} = \sum_{n=0}^{\infty} x^n = 1 + x + x^2 + x^3 + \cdots$

\subsection{常见导数公式}

$\frac{\mathrm{d}}{\mathrm{d}x}\tan x=\sec ^{2}x$

$\frac{\mathrm{d}}{\mathrm{d}x}\cot x=-\csc ^{2}x$

$\frac{\mathrm{d}}{\mathrm{d}x}\sec x=\sec x\tan x$

$\frac{\mathrm{d}}{\mathrm{d}x}\csc x=-\csc x\cot x$

$\frac{\mathrm{d}}{\mathrm{d}x}\arcsin x={\frac {1}{\sqrt {1-x^{2}}}}$

$\frac{\mathrm{d}}{\mathrm{d}x}\arccos x=-{\frac {1}{\sqrt {1-x^{2}}}}$

$\frac{\mathrm{d}}{\mathrm{d}x}\arctan x={\frac {1}{1+x^{2}}}$

\subsection{积分(省略 $+C$)}

$ \int {\frac {1}{x^{2}+\alpha ^{2}}}{\mbox{d}}x={\frac {\arctan {\frac {x}{\alpha }}}{\alpha }} $

$ \int {\frac {1}{\pm x^{2}\mp \alpha ^{2}}}{\mbox{d}}x={\frac {\ln \left({\frac {x\mp \alpha }{\pm x+\alpha }}\right)}{2\alpha }} $

$ \int {\frac {1}{ax^{2}+b}}{\mbox{d}}x={\frac {1}{\sqrt {ab}}}\arctan {\frac {{\sqrt {a}}x}{\sqrt {b}}} $

$ \int {\frac {1}{\sqrt {a^{2}-x^{2}}}}{\mbox{d}}x=\arcsin {\frac {x}{a}}=-\arccos {\frac {x}{a}} $

$ \int \sec ^{2}x{\mbox{d}}x=\tan x $

$ \int \csc ^{2}x{\mbox{d}}x=-\cot x $

$ \int \sec x\tan x{\mbox{d}}x=\sec x $

$ \int \csc x\cot x{\mbox{d}}x=-\csc x $

$ \int \tan x{\mbox{d}}x=-\ln {\left|\cos {x}\right|}=\ln {\left|\sec x\right|} $

$ \int \cot x{\mbox{d}}x=\ln {\left|\sin x\right|} $

$ \int \sec x{\mbox{d}}x=\ln {\left|\sec x+\tan x\right|} $

$ \int \csc x{\mbox{d}}x=\ln {\left|\csc x-\cot x\right|}=\ln {\left|{\tan x-\sin x \over \sin x\tan x}\right|} $

$ \int x^{n}e^{ax}{\mbox{d}}x={\frac {1}{a}}x^{n}e^{ax}-{\frac {n}{a}}\int x^{n-1}e^{ax}{\mbox{d}}x $

$ \int _{-\infty }^{\infty }e^{-ax^{2}+bx+c}\,\mathrm{d}x={\sqrt {\frac {\pi }{a}}}\,e^{{\frac {b^{2}}{4a}}+c} $


\section{Probability}

\subsection{Distribution}

\subsubsection{Poisson Distribution}

记号:$X \sim \text{Poisson}(\lambda)$

概率:$p(k)=\frac{\lambda^k}{k!}e^{-\lambda}$

本质上是一个$n\to\infty$的二项分布,$\lambda=np$。

性质:$\mathbb{E}(X)=\lambda,\text{Var}(X)=\lambda$

Approximate Bin: $n$ large, $p$ small ($n \ge 50, np \le 5$)

\subsubsection{Hypergeometric Distribution}

记号:$X \sim \text{Hypergeomet}(n, N, m)$

概率:$p(k)=\frac{\binom{m}{k}\binom{N-m}{n-k}}{\binom{N}{n}}$

$N$ 个球,$m$ 个红球,不放回取出 $n$ 个,有 $k$ 红球。

$\mathbb{E}(X)=n\cdot\frac{m}{N}, \mathrm{Var}(X)=n\cdot \frac{m}{N}\left(1-\frac{m}{N}\right)\left(1-\frac{n-1}{N-1}\right)$

\subsubsection{Normal Distribution}

PDF:$f(x)=\frac{1}{\sigma\sqrt{2\pi}}e^{-\frac{\left(x-\mu\right)^2}{2\sigma^2}}$

To $N(0, 1)$: $Z = \frac{X - \mu}{\sigma}$

Approximate Bin: $np(1-p)\ge 10$

$Z\sim N(0, 1), \mathbb{E}(g'(Z))=\mathbb{E}(Zg(Z))$,  assuming that $\lim_{x\to \infty}\frac{g(x)}{e^{\frac{x^2}{2}}}=0$. So $\mathbb{E}(Z^{n+1})=n\mathbb{E}(Z^{n-1})$.

\subsubsection{Exponential Distribution}

$X\sim \mathrm{Exp}(\lambda)$

PDF:$f(x)=\lambda e^{-\lambda x}, x\ge 0$

CDF:$F(X)=1-e^{-\lambda x}, x\ge 0$

$\mathbb{P}r(X>x)=e^{-\lambda x}, x\ge 0$

$\mathbb{E}(X^n)=\frac{n}{\lambda}\mathbb{E}(X^{n-1})=\frac{n!}{\lambda^n}$

$\mathbb{E}(X)=\frac{1}{\lambda}, \mathrm{Var}(X)=\frac{1}{\lambda^2}$

\subsubsection{Gamma Distribution}

考试都用 $\Gamma(\alpha, \beta)$ 的形式

$\Gamma(x)=\int_{0}^{\infty}u^{x-1}e^{-u}\mathrm{d}u, x>0$

$\Gamma(x)=(x-1)\Gamma(x - 1)$

$\Gamma(n)=(n-1)!$

PDF:$f(x)=\frac{\lambda^\alpha}{\Gamma(a)}x^{\alpha - 1}e^{-\lambda x}, x>0$

$\alpha$:发生次数

$\mathbb{E}(X)=\frac{\alpha}{\lambda}, \mathrm{Var}(X)=\frac{\alpha}{\lambda^2}$

$\mathbb{E}(X^n)=\frac{n+\alpha-1}{\lambda}\mathbb{E}(X^{n-1})=\frac{\alpha^{\overline{n}}}{\lambda^n}$

\subsubsection{Chi-Squared Distribution}

$X\sim \chi^2(k)$

PDF:$f(x)=\frac{1}{2^{\frac{k}{2}}\Gamma\left(\frac{k}{2}\right)}x^{\frac{k}{2}-1}e^{-\frac{x}{2}}$

$k$(自由度)个 $N(0, 1)$所组成向量长度平方的分布。

$\mathbb{E}(X)=k$

\subsection{MGF}

$M(X)=\mathbb{E}\left[e^{tX}\right]$

$M^{(m)}(X)=\mathbb{E}\left[X^m\right]$

\subsection{Covariance}

$\mathbb{E}(g(X)\cdot h(Y))=\mathbb{E}(g(X))\cdot \mathbb{E}(h(Y))$

$\mathrm{Cov}(X, Y)=\mathbb{E}((X-\mu_X)(Y-\mu_Y))=\mathbb{E}(XY)-\mathbb{E}(X)\mathbb{E}(Y)$

$X, Y$ 独立 $\to$ $\mathrm{Cov}(X, Y)=0$,但反过来不行

$\mathrm{Cov}$的线性性:

$\mathrm{Cov}\left(a+\sum_{i=1}^{n}b_iX_i, c+\sum_{i=1}^{m}d_iY_i\right)=\sum_{i=1}^{n}\sum_{j=1}^{m}b_id_j\mathrm{Cov}(X_i, Y_j)$

$\mathrm{Var}\left(a+\sum_{i=1}^{n}b_iX_i\right)=\sum_{i=1}^{n}\sum_{j=1}^{n}b_ib_j\mathrm{Cov}(X_i, Y_j)$

Correlation coefficient: $\rho=\frac{\mathrm{Cov}(X, Y)}{\sqrt{\mathrm{Var}(X)\mathrm{Var}(Y)}}=\frac{\sigma_{XY}}{\sigma_X\sigma_Y}$

柯西不等式:$-1\le\rho\le 1$,$\rho=\pm 1$ 当且仅当 $X, Y$ 线性相关

\subsection{Sample Mean/Variance}

$\overline{X}=\frac{1}{n}\sum_{i=1}^{n}X_i$

$S^2=\frac{1}{n-1}\sum_{i=1}^{n}\left(X_i-\overline{X}\right)^2$

$\frac{1}{n-1}$ 的原因:为了对齐 $\sigma^2$,$\mathbb{E}(S^2)=\sigma^2$

\subsection{Conditioning}

$\mathbb{P}\mathrm{r}(E)=\int_{\mathbb{R}}\mathbb{P}\mathrm{r}(E\mid X=x)p(x)\mathrm{d}x$

$\mathbb{E}(Y)=\mathbb{E}_X[\mathbb{E}_Y(Y\mid X)]$

$\mathrm{Var}(Y\mid X)=\mathbb{E}(Y^2\mid X)-[\mathbb{E}(Y\mid X)]^2$

$\mathrm{Var}(Y)=\mathrm{Var}[\mathbb{E}(Y\mid X)]+\mathbb{E}[\mathrm{Var}(Y\mid X)]$

$T=\sum_{i=1}^{N}X_i$,$X, N$ 随机

$Var(T)=(\mathbb{E}(X))^2\mathrm{Var}(N)+\mathbb{E}(N)\mathrm{Var}(X)$

\subsection{Central Limit Theorem}

\subsubsection{Inequalities}

Markov's inequality: $\mathbb{P}\mathrm{r}\left\{X\ge t\right\}\le\frac{\mathbb{E}(X)}{t}$,要求是 $X\ge 0, t>0$

Chebyshev's Inequality:$\mathbb{P}\mathrm{r}\{|X-\mathbb{E}(X)|\ge t\}\le\frac{\mathrm{Var}(X)}{t^2}$

$\mathbb{P}\mathrm{r}\{|X-\mathbb{E}(X)|\ge k\sigma\}\le\frac{1}{k^2}$

\subsubsection{Law of Large Numbers}

$\overline{X}_n=\frac{1}{n}\sum_{i=1}^{n}X_i$

Weak: $\lim_{n\to\infty}\mathbb{P}\mathrm{r}\left\{\left|\overline{X}_n-\mu\right|>\epsilon\right\}=0$

Strong: $\mathbb{P}\mathrm{r}\left\{\lim_{n\to\infty}\overline{X}_n=\mu\right\}=1$

\subsubsection{CLT}

$S_n=\sum_{i=1}^{n}X_i$

$\lim_{n\to\infty}\mathbb{P}\mathrm{r}\left\{\frac{S_n-n\mu}{\sigma\sqrt{n}}\le x\right\}=\Phi(x)$

\end{multicols}

\end{document}
